\subsection{Elevation}
\subsection*{Defintion}
Le dataset d’élévation \textbf{be15\_grd} est constitué d’un seul raster, où la valeur de chaque pixel représente l’élévation à cet emplacement.

Ce raster est structuré sous la forme d’un dossier contenant plusieurs fichiers \textbf{.adf}, qui composent ensemble un raster au format ESRI Grid couvrant toute la carte du monde.

Pour convertir les coordonnées d’un pixel en longitude et latitude, nous  utilisons les systèmes de référence définis par la norme EPSG, notamment l’EPSG:4326 (WGS84), qui décrit la manière standard de représenter les coordonnées géographiques sur la Terre en longitude et latitude.

\subsection*{Chargement des données}
 Nous utilisons les bibliothèques suivantes pour charger et manipuler les datasets de l'élevation :
 \begin{itemize}
    \item \textbf{rioxarray :} pour ouvrir et manipuler les fichiers raster.
    \item \textbf{geopandas :} pour ouvrir les fichiers de forme des pays et extraire les contours, ainsi que pour sélectionner les zones correspondant à l’Algérie et à la Tunisie.
    \item \textbf{matplotlib :} pour la visualisation des graphes d’analyse.
 \end{itemize}
Ainsi, comme pour le dataset du sol, nous obtenons au final une liste des valeurs d’élévation pour chaque point en Algérie et en Tunisie.

\begin{figure}[H]
    \centering
    \includegraphics[width=0.8\linewidth]{elevation/map.png}
    \caption{Carte de l'élévation de l'Algérie et la Tunisie}
    \label{fig:elevation-map}
\end{figure}

\subsection*{Prétraitement de analyse de données}
Comme il n y a pas de données manquante ou abberantes, nous passons directement au prétraitement et analyse :

\begin{itemize}
    \item \textbf{Analyse univaiée :} calcul des statestiques descriptives pour l'attribue de l'élévation.
    
    \begin{table}[H]
        \centering
        \begin{tabular}{l r}
        \hline
        \textbf{Statistique} & \textbf{Valeur} \\
        \hline
        Count & 13\,183\,993 \\
        Mean  & 536.37 \\
        Std   & 325.48 \\
        Min   & -872 \\
        25\%  & 310 \\
        50\%  & 463 \\
        75\%  & 697 \\
        Max   & 2877 \\
        \hline
        \end{tabular}
        \caption{Statistiques descriptives des valeurs d’élévation.}
    \end{table}

    \item \textbf{Distribution de données :} visualisation de la distribution des valeurs de l'élévation  à l’aide d’histogrammes et de boites à moustache.

    \begin{figure}[H]
        \centering
        \includegraphics[width=1\linewidth]{elevation/histogram.png}
        \caption{Histogramme des valeurs d'élévation}
        \label{fig:elevation-histograms}
    \end{figure}
    \begin{figure}[H]
        \centering
        \includegraphics[width=1\linewidth]{elevation/boxplot.png}
        \caption{Boite à moustache des valeurs d'élévation}
        \label{fig:elevation-boxplots}
    \end{figure}


\end{itemize}