\subsection{Sampling}
Pour gérer le déséquilibre entre les classes \texttt{fire} et \texttt{no-fire} et améliorer la représentativité spatiale des données, deux méthodes ont été appliquées.

\subsection*{1. Sous-échantillonnage de la classe majoritaire}
\begin{itemize}
    \item La classe \texttt{no-fire} étant largement majoritaire, un sous-échantillonnage aléatoire a été effectué pour obtenir un ratio contrôlé de 1:3 (\texttt{fire:no-fire}).
    \item Cette approche permet de réduire le risque de biais des modèles tout en conservant une dominance réaliste de la classe majoritaire.
\end{itemize}

\subsection*{2. Binning spatial}
\begin{itemize}
    \item Les points proches dans la grille ont été regroupés pour éviter la redondance spatiale et uniformiser l’échantillonnage.
    \item Chaque cellule de 0.01° (~1,1 km) a conservé un seul point représentatif, garantissant une couverture spatiale homogène.
\end{itemize}

\subsection*{Impact sur le dataset final}
\begin{itemize}
    \item Déséquilibre initial de ~1:99 (fire:no-fire) réduit à 1:3.
    \item Réduction de la sur-représentation de la classe \texttt{no-fire}.
    \item Optimisation du nombre de points et du temps de calcul.
    \item Validation visuelle confirmant une distribution spatiale équilibrée des points \texttt{fire} et \texttt{no-fire}.
\end{itemize}

\begin{figure}[htbp]
    \centering
    \includegraphics[width=0.7\textwidth]{fire/sampling.png}
    \caption{dataset fire apres sampling.}
    \label{fig:une_image}
\end{figure}
