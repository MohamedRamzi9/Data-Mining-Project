\subsection{Land Cover}

\subsection*{Définition}
Le dataset utilisé se compose de deux jeux de données distincts correspondant aux versions vectorielles nationales du produit GlobCover (2005) pour l’Algérie et la Tunisie, développé initialement par l’Agence Spatiale Européenne (ESA).

Cette étude porte sur l’analyse des données d’occupation des sols (Land Cover) pour ces deux régions géographiques. L’objectif principal est de manipuler des données géospatiales au format Shapefile afin d’extraire des informations statistiques pertinentes et de préparer un jeu de données structuré et propre, destiné à des analyses ultérieures ou à des applications de visualisation.

\subsection*{Lecture des données}
La première étape a consisté à importer les bibliothèques nécessaires au traitement de données 
\begin{itemize}
    \item \textbf{GeoPandas} : manipulation et analyse de données géospatiales vectorielles.
    \item \textbf{Pandas} : gestion, fusion et analyse des données tabulaires.
    \item \textbf{NumPy} : calculs numériques et transformations mathématiques.
    \item \textbf{Matplotlib} : visualisation graphique et cartographique.
    \item \textbf{Seaborn} : visualisations statistiques avancées.
    \item \textbf{SciPy} : analyses statistiques et tests de normalité.
    \item \textbf{scikit-learn} : encodage des variables catégorielles.
\end{itemize}


Le système de coordonnées de référence (CRS) a été vérifié pour les deux jeux de données et est défini sur EPSG:4326 (WGS 84).

\subsection*{Nettoyage des données}
Une analyse initiale a permis d’identifier des incohérences et des valeurs manquantes dans plusieurs colonnes clés :  

\begin{itemize}
    \item Algérie : valeurs nulles dans les champs \texttt{AREA} et \texttt{LCCCODE}.
    \item Tunisie : valeurs nulles dans les champs \texttt{AREA\_M2} et \texttt{LCCCode}.
\end{itemize}

Pour assurer l’homogénéité des données :  

\begin{itemize}
    \item Les colonnes de superficie ont été unifiées sous le nom \texttt{area\_sqm}.
    \item Les codes de classification de l’occupation du sol ont été harmonisés sous \texttt{lcc\_code}.
\end{itemize}

Les shapefiles ont ensuite été fusionnés en un seul jeu de données couvrant l’ensemble de la zone d’étude, suivi d’une suppression des doublons.  



\subsection*{Prétraitement et analyse des données}
Pour l’attribut \texttt{area\_sqm}, un traitement des valeurs aberrantes a été appliqué :  
\begin{itemize}
    \item Transformation logarithmique pour réduire l’asymétrie des distributions.
    \item Détection et traitement des valeurs extrêmes via la méthode de l’écart interquartile (IQR).
\end{itemize}

de plus, une correction géométrique a été réalisée pour garantir la validité topologique des entités spatiales.
\begin{itemize}
    \item \textbf{Analyse univariée :} calcul des indicateurs principaux (moyenne, médiane, minimum, maximum) pour chaque variable numérique.
    \item \textbf{Repérage des anomalies :} identification des valeurs inhabituelles et visualisation à l’aide d’histogrammes et de boîtes à moustaches.
\end{itemize}

\begin{table}[htbp]
\centering
\begin{tabular}{lccc}
\hline
 & log\_area\_sqm & GRIDCODE & pays \\
\hline
count & 438513.0 & 438513.0 & 438513 \\
unique & - & - & 2 \\
top & - & - & Algérie \\
freq & - & - & 386454 \\
mean & 12.9033 & 156.2099 & - \\
std & 0.9660 & 65.6615 & - \\
min & 11.5152 & 14.0 & - \\
25\% & 12.0597 & 150.0 & - \\
50\% & 12.6450 & 200.0 & - \\
75\% & 13.4245 & 200.0 & - \\
max & 15.4718 & 210.0 & - \\
\hline
\end{tabular}
\caption{Statistiques descriptives des variables.}
\label{tab:descriptives}
\end{table}

\begin{figure}[htb]
    \centering
    \includegraphics[width=0.8\textwidth]{land_cover_dataset/boxplot.png}
    \caption{boxplot}
    \label{fig:photo1}
\end{figure}
\begin{figure}[htb]
    \centering
    \includegraphics[width=0.8\textwidth]{land_cover_dataset/q-q.png}
    \caption{q-q plots}
    \label{fig:photo1}
\end{figure}
\begin{figure}[htb]
    \centering
    \includegraphics[width=0.8\textwidth]{land_cover_dataset/sactter plots.png}
    \caption{scatter plot}
    \label{fig:photo1}
\end{figure}
\begin{figure}[htb]
    \centering
    \includegraphics[width=0.8\textwidth]{land_cover_dataset/area.png}
    \caption{distribution de log\_area }
    \label{fig:photo1}
\end{figure}
\begin{figure}[htb]
    \centering
    \includegraphics[width=0.8\textwidth]{land_cover_dataset/gridcode.png}
    \caption{distribution de gridcode}
    \label{fig:photo1}
\end{figure}
\begin{figure}[htb]
    \centering
    \includegraphics[width=0.8\textwidth]{land_cover_dataset/lcccode.png}
    \caption{distribution de lcccode }
    \label{fig:photo1}
\end{figure}
\begin{figure}[htb]
    \centering
    \includegraphics[width=0.8\textwidth]{land_cover_dataset/matrice.png}
    \caption{matrice de correlation}
    \label{fig:photo1}
\end{figure}




