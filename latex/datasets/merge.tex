\subsection{Fusionnement des datasets}

À la fin du traitement et du nettoyage des différents datasets, il est nécessaire de les combiner en un seul dataset contenant l’ensemble des attributs de chacun, afin de l’utiliser dans la phase d’entraînement.
Le fusionnement se fait sur les attributs longitude et latitude, car ce sont les informations communes qui permettent d’identifier et de relier les mêmes points géographiques entre les différents jeux de données.

Pour réaliser cette fusion, nous appliquons une jointure à gauche (left join) entre le dataset des feux et les autres datasets.
Ce choix garantit que tous les points liés aux feux sont conservés, et qu’aucune observation contenant la valeur à prédire — l’existence ou non d’un feu — n’est perdue lors de l’opération de fusion.


\subsection*{Fusionnement des données Fire  Land Cover}
Les points Fire ont été associés aux données climatiques via un \textbf{matching spatial par plus proche voisin} utilisant \texttt{cKDTree} de SciPy.  
Les données climatiques, organisées en grille régulière, ont été agrégées par saison : somme pour les précipitations (\texttt{log\_precip}) et moyenne pour les températures (\texttt{tmax}, \texttt{amplitude\_thermique}).  
Chaque point de feu a ainsi reçu les valeurs climatiques de la cellule la plus proche, assurant une correspondance spatiale précise et une couverture complète.

\subsection*{Fusionnement des données avec Climat}
L’intégration des données Land Cover a combiné \textbf{buffer spatial} et \textbf{KDTree}.  
Les points Fire+Climat ont été convertis en GeoDataFrame et projetés en EPSG:4326.  
Un buffer de 0.005° (~500 m) autour de chaque point a permis d’assigner les attributs des polygones Land Cover via un \texttt{spatial join}.  
Pour les points hors polygones, le land cover du polygone le plus proche a été attribué via KDTree.  
Les duplications ont été résolues en conservant uniquement la première correspondance, garantissant précision et couverture complète.

\subsubsection*{Fusionnement du dataset des feux avec le dataset du sol}
Pour cette opération, nous constatons que les valeurs de longitude et de latitude dans les deux datasets ne sont pas alignées sur la même grille spatiale. Par conséquent, une jointure classique reposant sur une comparaison directe de ces coordonnées ne peut pas fonctionner.

Afin d’associer chaque point de feu avec son point du sol correspondant, nous devons effectuer une jointure spatiale reposant sur la recherche du point le plus proche (nearest neighbor).
Une solution efficace consiste à utiliser la structure de données KDTree, un arbre multidimensionnel qui permet d’effectuer des recherches de proximité très rapides.

L’arbre est construit à partir des positions géographiques des points du dataset du sol. Ensuite, pour chaque point du dataset du feu, nous interrogeons le KDTree pour identifier le point du sol le plus proche.

À la fin du processus, nous obtenons un dataset fusionné dans lequel chaque observation de feu est enrichie par les attributs du point du sol spatialement le plus proche.


\subsubsection*{Fusionnement du dataset des feux avec le dataset d’élévation}

Cette opération est beaucoup plus simple, car le dataset d’élévation est enregistré sous forme de raster.
Ce format offre directement une méthode d’échantillonnage : en fournissant une position géographique (longitude et latitude), il est possible de récupérer la valeur du pixel du raster auquel ce point appartient.

Ainsi, pour chaque point du dataset des feux, il suffit d’utiliser cette fonction d’échantillonnage pour extraire la valeur d’élévation correspondante. À la fin du processus, chaque point de feu est enrichi par la valeur d’altitude issue du raster. 


