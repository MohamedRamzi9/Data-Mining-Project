\subsection{Climat}
\subsection*{Defintion }
Le jeu de données climatique utilisé provient de la base \textit{WorldClim}, une base de données climatique mondiale largement utilisée en sciences de l’environnement. Il fournit des variables climatiques mensuelles, notamment la température et les précipitations, à une résolution spatiale de 5 minutes d’arc (environ 10 km). Ces données sont issues de l’interpolation de mesures provenant de stations météorologiques à l’échelle mondiale et sont adaptées aux analyses climatiques, écologiques et géospatiales à grande échelle.
\subsection*{Chargement de données}
L’analyse repose sur plusieurs bibliothèques Python complémentaires.  
\begin{itemize}
    \item \textbf{GeoPandas} est utilisé pour lire le fichier Land Cover au format GeoPackage et servir de masque spatial pour le découpage des données climatiques.
    \item \textbf{rioxarray} constitue l’outil principal pour la lecture, le découpage spatial et la sauvegarde des rasters climatiques au format \texttt{.tif}.
    \item \textbf{xarray} permet la gestion de données raster multidimensionnelles (temps, espace) et l’ouverture simultanée de plusieurs fichiers climatiques.
    \item \textbf{Rasterio} est employé pour une lecture bas-niveau des rasters et l’accès aux métadonnées lors des analyses univariées.
    \item \textbf{Pandas} est utilisé pour la gestion des dates et la création de DataFrames destinés aux analyses statistiques.
\end{itemize}
Les rasters climatiques (précipitations, température minimale et maximale) sont découpés à l’aide d’un masque spatial issu du fichier Land Cover, afin de restreindre l’étude à la zone couvrant l’Algérie et la Tunisie.  


\subsection*{Netoyage de données }
Le nettoyage des données vise à garantir la qualité, la cohérence et la fiabilité du jeu de données climatique. Les étapes principales sont :

\begin{itemize}
    \item Suppression des colonnes techniques non pertinentes issues des formats raster, telles que \texttt{band} et \texttt{spatial\_ref}.
    \item Détection, quantification et élimination des valeurs manquantes pour éviter tout biais dans les analyses.
    \item Identification et suppression des doublons afin d’assurer l’unicité des observations.
    \item Standardisation des noms de colonnes spatiales (\texttt{x} renommé en \texttt{longitude} et \texttt{y} en \texttt{latitude}).
    \item Réinitialisation de l’index pour obtenir un DataFrame plat et directement exploitable.
\end{itemize}

\subsection*{Pretraitement et analyse des donnees}
Le prétraitement des données a pour objectif d’améliorer la distribution des variables et d’optimiser le jeu de données .  
Une transformation logarithmique est appliquée aux précipitations afin de réduire l’asymétrie de leur distribution.  
Une variable dérivée, appelée amplitude thermique, est créée à partir de la différence entre la température maximale et minimale pour mieux représenter la variabilité thermique.  
Une forte corrélation entre les variables de température est observée, conduisant à la suppression de la température minimale afin de réduire la multicolinéarité.  
Le jeu de données final est ainsi limité à trois variables pertinentes : les précipitations transformées, la température maximale et l’amplitude thermique.  
Les valeurs aberrantes sont détectées à l’aide de la méthode de l’IQR sans être supprimées, car elles peuvent correspondre à des événements climatiques extrêmes réels.
\\Les données sont ensuite explorées à travers des analyses univariées et bivariées.

\begin{table}[htbp]
\centering
\caption{Statistiques descriptives finales des variables climatiques}
\begin{tabular}{lccc}
\hline
Statistique & \texttt{log\_precip} & \texttt{tmax} & \texttt{amplitude\_thermique} \\
\hline
Count & 148,292 & 148,292 & 148,292 \\
Mean  & 1.937 & 27.828 & 13.428 \\
Std   & 1.035 & 8.223  & 2.213  \\
Min   & 0.000 & 5.250  & 5.500  \\
25\%  & 1.082 & 21.000 & 12.000 \\
50\%  & 1.825 & 28.000 & 13.750 \\
75\%  & 2.733 & 34.750 & 15.000 \\
Max   & 5.588 & 48.000 & 20.250 \\
\hline
\end{tabular}
\label{tab:statistiques_finales}
\end{table}
\begin{figure}[H]
    \centering
    \includegraphics[width=0.9\textwidth]{climat/boxplot.png}
    \caption{boxplot.}
    \label{fig:boxplot}
\end{figure}
\begin{figure}[H]
    \centering
    \includegraphics[width=0.7\textwidth]{climat/matrice.png}
    \caption{matrice de correlation.}
    \label{fig:une_image}
\end{figure}

