\subsection{Soil}
\subsection*{Definition}
Le dataset du sol est constitué d’une base de données contenant plusieurs tables. La table la plus importante est la table \textbf{layers}, qui contient des enregistrements sur les propriétés des différentes couches de sol dans les zones d’Algérie et de Tunisie.

Chaque enregistrement couvre une zone et comporte plusieurs attributs, parmi lesquels :

\begin{itemize}
    \item \textbf{HWSD2\_SMU\_ID} l'identifiant de l'enregistrement
    \item le niveau de décomposition
    \item le pourcentage de carbone
    \item le pH de l’eau
    \item la texture du sol (sable, limon, argile)
    \item la teneur en humidité
    \item la densité apparente
    \item et d’autres caractéristiques physico-chimiques du sol
\end{itemize}

Cette table permet donc d’analyser la composition et la qualité des sols selon différentes régions et couches. Dans notre étude, nous nous interessons au niveau \textbf{D1} de la decompostions.

Il existe également un fichier HWSD2.bil, qui est un raster couvrant la carte du monde, où chaque pixel représente un point appartenant à une zone de sol spécifique.

Nous utilisons un troisième dataset contenant des fichiers de forme des pays afin de récupérer uniquement les identifiants des sols correspondant à l’Algérie et à la Tunisie. Ensuite, nous combinons tous les points du raster avec leurs enregistrements correspondants en utilisant l’identifiant SMU comme clé de jointure.

À la fin de ce processus, nous obtenons un dataset complet des propriétés du sol pour chaque point situé en Algérie et en Tunisie.


\subsection*{Chargement des données}
Nous utilisons les bibliothèques suivantes pour charger et manipuler le dataset du sol :

\begin{itemize}
\item \textbf{pyodbc :} pour le chargement des tables depuis la base de données.
\item \textbf{rasterio et rioxarray :} pour ouvrir et manipuler les fichiers raster.
\item \textbf{geopandas :} pour ouvrir les fichiers de forme des pays et extraire les contours, ainsi que pour sélectionner les zones correspondant à l’Algérie et à la Tunisie.
\item \textbf{pandas et numpy :} pour la manipulation des données, le prétraitement et l’analyse.
\item \textbf{matplotlib :} pour la visualisation des graphes d’analyse.
\end{itemize}

À l’issue de ce processus, nous obtenons un dataset complet des propriétés du sol pour chaque point situé en Algérie et en Tunisie.

\subsection*{Nettoyage des données}

Le nettoyage des données a été réalisé afin de préparer le dataset pour l’analyse. Les étapes principales incluent :

\begin{itemize}
    \item Pour les attributs numériques, les valeurs manquantes sont codées par des valeurs négatives et sont remplacées par la moyenne des points adjacents selon les coordonnées longitude et latitude.
    \item Pour les attributs catégoriques, les valeurs manquantes sont représentées par le caractère ‘-’ et sont sont remplacés par le mode des points voisinages.
\end{itemize}
\begin{figure}[H]
    \centering
    \includegraphics[width=0.8\textwidth]{soil/missing-values-before.png} % Path to your image
    \includegraphics[width=0.8\textwidth]{soil/missing-values-after.png}
    \caption{Nombres de valeurs manquante avant et aprés le netoyage.}
    \label{fig:soil-valeur-manquante}
\end{figure}

    
\subsection*{Prétraitement et analyse des données}

Après le nettoyage, les données ont été prétraitées et analysées selon les étapes suivantes :

\begin{itemize}
    \item Analyse univariée : calcul des statistiques descriptives pour chaque attribut numérique, telles que la moyenne, la médiane, le minimum et le maximum...
    
    \begin{table}[htbp]
    \centering
    \caption{Statistiques descriptives des attributs numériques (arrondies à 2 décimales)}
    \begin{tabular}{lrrrrrrrrr}
    \hline
    Attribut & Count & Mean & Std & Min & 25\% & 50\% & 75\% & Max \\
    \hline
    COARSE & 858.0 & 12.08 & 9.69 & 2.0 & 4.0 & 9.0 & 18.0 & 46.0 \\
    SAND & 858.0 & 47.95 & 14.23 & 13.0 & 40.42 & 47.0 & 55.0 & 90.0 \\
    SILT & 858.0 & 30.22 & 7.6 & 5.0 & 27.0 & 30.0 & 36.0 & 53.0 \\
    CLAY & 858.0 & 21.74 & 9.67 & 4.0 & 16.0 & 20.0 & 24.0 & 55.0 \\
    TEXTURE\_USDA & 858.0 & 8.95 & 2.4 & 3.0 & 9.0 & 9.0 & 11.0 & 12.0 \\
    BULK & 858.0 & 1.43 & 0.12 & 0.12 & 1.37 & 1.44 & 1.48 & 1.76 \\
    REF\_BULK & 858.0 & 1.71 & 0.13 & 1.2 & 1.65 & 1.71 & 1.78 & 2.04 \\
    ORG\_CARBON & 858.0 & 0.93 & 0.67 & 0.24 & 0.59 & 0.7 & 1.16 & 7.33 \\
    PH\_WATER & 858.0 & 7.68 & 0.7 & 3.66 & 7.4 & 8.0 & 8.2 & 8.6 \\
    TOTAL\_N & 858.0 & 0.96 & 0.43 & 0.22 & 0.73 & 0.88 & 1.12 & 3.69 \\
    CN\_RATIO & 858.0 & 9.98 & 1.52 & 7.4 & 9.0 & 9.0 & 10.0 & 24.0 \\
    CEC\_SOIL & 858.0 & 15.75 & 7.11 & 4.0 & 13.0 & 15.0 & 17.0 & 41.0 \\
    CEC\_CLAY & 858.0 & 59.29 & 14.67 & 16.0 & 48.0 & 61.0 & 71.0 & 83.0 \\
    CEC\_EFF & 858.0 & 36.81 & 22.98 & 3.0 & 25.0 & 31.0 & 38.0 & 143.0 \\
    TEB & 858.0 & 35.82 & 23.65 & 2.0 & 21.0 & 31.0 & 38.0 & 143.0 \\
    BSAT & 858.0 & 93.23 & 11.79 & 10.0 & 93.0 & 99.0 & 100.0 & 100.0 \\
    ALUM\_SAT & 858.0 & 0.58 & 3.46 & 0.0 & 0.0 & 0.0 & 0.0 & 40.0 \\
    ESP & 858.0 & 6.92 & 13.52 & 0.88 & 2.0 & 3.0 & 4.0 & 67.0 \\
    TCARBON\_EQ & 858.0 & 9.31 & 6.69 & 0.0 & 4.5 & 8.9 & 12.9 & 31.3 \\
    GYPSUM & 858.0 & 4.45 & 10.97 & 0.0 & 0.3 & 0.6 & 3.3 & 57.6 \\
    ELEC\_COND & 858.0 & 2.19 & 4.43 & 0.0 & 1.0 & 1.0 & 1.0 & 32.0 \\
    \hline
    \end{tabular}
    \end{table}

    \item Détection des valeurs aberrantes : identification des valeurs extrêmes et visualisation à l’aide d’histogrammes et de boites à moustache.

    \begin{figure}[H]
        \centering
        \includegraphics[width=1\linewidth]{soil/histograms.png}
        \caption{Histogramme des valeurs numériques}
        \label{fig:soil-histograms}
    \end{figure}
    \begin{figure}[H]
        \centering
        \includegraphics[width=1\linewidth]{soil/boxplots.png}
        \caption{Boite à moustache des valeurs numériques}
        \label{fig:soil-boxplots}
    \end{figure}

    

    \item Analyse de corrélation : calcul et affichage de la matrice de corrélation avant et après suppression des valeurs fortement corrélées.
    
    \begin{figure}[H]
        \centering
        \includegraphics[width=1\linewidth]{soil/correlation-matrix.png}
        \caption{Matrice de corrélation}
        \label{fig:soil-correlation-matrix}
    \end{figure}
    
    \item Exportation des données nettoyées : sauvegarde du dataset final sous forme de fichier .parquet.
\end{itemize}