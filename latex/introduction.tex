\chapter{Introduction}
Les incendies de forêt constituent un défi environnemental et socio-économique majeur, entraînant la perte de végétation, la dégradation des sols et de graves dommages écologiques. Dans le cadre de ce projet de Data Mining, l'objectif est de développer un modèle prédictif capable d'anticiper l'occurrence des feux en s'appuyant sur des données environnementales riches et variées. La zone d'étude se concentre sur l'Algérie et la Tunisie pour l'année 2024, en fusionnant des données provenant de sources spatiales et terrestres de référence.
Le projet adopte une approche structurée en trois phases : l'analyse et le prétraitement des données, l'application d'algorithmes d'apprentissage supervisé pour la prédiction, et l'utilisation de l'apprentissage non-supervisé (clustering) pour identifier des zones à haut risque. Cette démarche vise non seulement à atteindre une précision technique, mais aussi à fournir des outils d'aide à la décision pour les stratégies de prévention et de gestion des incendies.
\section*{problématique}
Face à l'intensification des risques climatiques, la question centrale de cette étude est la suivante : Comment l'intégration et l'analyse de données hétérogènes (climatiques, pédologiques et géospatiales) peuvent-elles permettre de modéliser et de prédire avec précision l'occurrence des incendies de forêt en Algérie et en Tunisie ?.
Cette problématique se décline en plusieurs défis techniques et scientifiques :
\begin{itemize}
    \item  L'influence des variables : Dans quelle mesure les caractéristiques du sol (humidité, carbone organique, texture USDA) et les facteurs climatiques (température, précipitations) interagissent-ils pour favoriser un départ de feu ?.
    \item  Le défi des données : Comment traiter et fusionner efficacement des jeux de données aux résolutions et formats disparates (rasters d'élévation, shapefiles d'occupation du sol de la FAO, points de feu de la NASA) ?
\end{itemize}