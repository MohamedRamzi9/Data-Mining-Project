\chapter{Conclusion}
Ce projet a permis de poser les bases d'un système robuste de prédiction des risques d'incendie pour la région Algérie-Tunisie. La première phase cruciale de prétraitement et d'intégration a permis d'harmoniser des données complexes, allant de la texture du sol à 20 cm de profondeur (couche D1 de la base HWSD) jusqu'aux variables climatiques de WorldClim. Le recours à des techniques avancées comme les KDTrees a assuré une fusion spatiale précise malgré des grilles de coordonnées divergentes.
L'application future des modèles de Supervised Learning et d'Unsupervised Learning permettra de transformer ces données brutes en une cartographie prédictive opérationnelle. En identifiant les zones de vulnérabilité et les schémas naturels de regroupement des feux, ce travail fournit des perspectives interprétables essentielles pour la conception de systèmes d'alerte précoce et de plans de prévention des feux de forêt. En conclusion, la maîtrise du cycle complet de la donnée, de l'extraction à la modélisation "from scratch", renforce la fiabilité des outils de surveillance environnementale pour ces deux pays